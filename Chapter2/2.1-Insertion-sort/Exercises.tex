2.1-2
Rewrite the INSERTION-SORT procedure to sort into nonincreasing instead of nondecreasing odrder.

Answer:
NONDECREASING-INSERTION-SORT (A)
  for j = 2 to A.length
    key = A[j]
    // Insert A[j] into the sorted sequence A[1 ... j-1]
    i = j - 1
    while i > 0 and A[i] < key
      A[i+1] = A[i]
      i = i - 1
    A[i+1] = key
    
2.1-3
Consider the searching problem:
  Input: A sequence of n numbers A = <a_1, a_2, ..., a_n> and a value v.
  Output: An index i such that v = A[i] or the special value NIL if v does not apprear in A.
Write pseudocode dor linear search, which scans through the sequence, looking for v. Using a loop invariant, prove that your algorithm is correct. Make sure that your loop invariant fulfills the threee necessary properties.

Answer:
LINEAR-SEARCH(A, v)
  for i = 1 to A.lengh
    if A[i] == key
      return i
  return NIL
  
loop invariant: It has not found i.
  
2.1-4
Consider the problem of adding two n-bit binary intergers, stored in two n-element arrays A and B. The sum of the two integers should be strored in binary form in an (n+1)-element array C. State the problem formally and write pseudocode for adding the two integers.

Answer:
BINARY-ADD(A, B)
  carry = 0
  for i = A.length to 1
    if carry == 0, A[i] == 0, B[i] == 0
      carry = 0
      C[i+1] = 0
    else if exactly one of carry, A[i], B[i] is 1
      carry = 0
      C[i+1] = 1
    else if exactly two of carry, A[i], B[i] are 2
      carry = 1
      C[i+1] = 0
    else 
      carry = 1
      C[i+1] = 1
  C[1] = carry
  return C

      
