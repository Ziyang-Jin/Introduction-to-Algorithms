\documentclass{article}
\usepackage{amsmath}
\usepackage{algorithm}
\usepackage[noend]{algpseudocode}
\title{Chapter 2 Exercises}
\author{Ziyang Jin}
\date{May 2016}
\setlength{\parindent}{0em}
\setlength{\parskip}{1em}

\begin{document}
2.1-2
Rewrite the INSERTION-SORT procedure to sort into nonincreasing instead of nondecreasing odrder.

Answer:
\begin{algorithm}
\begin{algorithmic}[1]
\Procedure{NON-DEREASING-INSERTION-SORT}{A}
\For{\texttt{j = 2 to A.length}}
    \State key = A[j]
    \State i = j - 1
    \While{\texttt{i > 0 and A[i] < key}}
        \State A[i+1] = A[i]
        \State i = i - 1
    \EndWhile
    \State A[i+1] = key
\EndFor
\EndProcedure
\end{algorithmic}
\end{algorithm}
    
2.1-3
Consider the searching problem:
  Input: A sequence of n numbers $A = <a_1, a_2, ..., a_n>$ and a value v.
  Output: An index i such that v = A[i] or the special value NIL if v does not apprear in A.
Write pseudocode dor linear search, which scans through the sequence, looking for v. Using a loop invariant, prove that your algorithm is correct. Make sure that your loop invariant fulfills the threee necessary properties.

Answer:
LINEAR-SEARCH(A, v)
  for i = 1 to A.lengh
    if A[i] == key
      return i
  return NIL

\begin{algorithm}
\begin{algorithmic}[1]
\Procedure{LINEAR-SEARCH}{A,v}
\For{texttt{i = 1 to A.lengh}}
    \If{A[i] == key}
        \State return i
    \EndIf
\EndFor
\State return NIL
\EndProcedure
\end{algorithmic}
\end{algorithm}
  
loop invariant: It has not found i.
  
2.1-4
Consider the problem of adding two n-bit binary intergers, stored in two n-element arrays A and B. The sum of the two integers should be strored in binary form in an (n+1)-element array C. State the problem formally and write pseudocode for adding the two integers.

Answer:

\begin{algorithm}
\begin{algorithmic}[1]
\Procedure{BINARY-ADD}{A, B}
\State carry = 0
\For{texttt{i = A.length to 1}}
    \If{none of carry, A[i], B[i] is 1}
        \State carry = 0
        \State C[i+1] = 0
    \ElsIf{exactly one of carry, A[i], B[i] is 1}
        \State carry = 0
        \State C[i+1] = 1
    \ElsIf{exactly two of carry, A[i], B[i] are 1}
       \State carry = 1
       \State C[i+1] = 0
    \Else
        \State carry = 1
        \State C[i+1] = 1
    \EndIf
\EndFor
\State C[1] = carry
\State return C
\EndProcedure
\end{algorithmic}
\end{algorithm}

\end{document}
