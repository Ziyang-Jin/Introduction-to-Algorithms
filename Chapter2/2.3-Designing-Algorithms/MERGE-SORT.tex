\documentclass{article}
\usepackage{amsmath}
\usepackage{algorithm}
\usepackage{mathtools}
\usepackage[noend]{algpseudocode}
\title{Merge Sort}
\author{Ziyang Jin}
\date{May 2016}
\setlength{\parindent}{0em}
\setlength{\parskip}{1em}

\begin{document}

\begin{algorithm}
\begin{algorithmic}[1]
\Procedure{MERGE}{A, p, q, r}
\State $n_1$ = q - p + 1
\State $n_2$ = r - p
\State let L[1 ... $n_1$+1] and R[1 ... $n_2$+1] be new arrays
\For{i = 1 to $n_1$}
    \State L[i] = A[p+i-1]
\EndFor
\For{j = 1 to $n_2$}
    \State R[j] = A[q+j]
\EndFor
\State L[$n_1$+1] = $\infty$
\State R[$n_2$+1] = $\infty$
\State i = 1
\State j = 1
\For{k = p to r}
    \If{L[i] $<=$ R[j]}
        \State A[k] = L[i]
        \State i = i + 1
    \Else
        \State A[k] = R[j]
        \State j = j + 1
    \EndIf
\EndFor
\EndProcedure
\end{algorithmic}
\end{algorithm}  


\begin{algorithm}
\begin{algorithmic}[1]
\Procedure{MERGE-SORT}{A, p, r}
\If{p $<$ r}
    \State q = $\lfloor{(p+r)/2}\rfloor$
    \State MERGE-SORT(A, p, q)
    \State MERGE-SORT(A, q+1, r)
    \State MERGE(A, p, q, r)
\EndIf
\EndProcedure
\end{algorithmic}
\end{algorithm}

\end{document}
