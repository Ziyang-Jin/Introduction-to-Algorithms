\documentclass{article}
\usepackage{amsmath}
\usepackage{algorithm}
\usepackage[noend]{algpseudocode}
\title{Chapter 2 Exercises}
\author{Ziyang Jin}
\date{May 2016}
\setlength{\parindent}{0em}
\setlength{\parskip}{1em}

\begin{document}
2.3-6
Observe that the while loop of lines 5-7 of the INSERTION-SORT procedure in Section 2.1 uses a linear search to scan (backward) through the sorted subarray A[1 ... j-1]. Can we use a binary search instead to improve the overall worst-case running time of insertion sort to $\Theta (n lg n)$?

Answer:
We can use binary search to find the place to insert in $\Theta (lg n)$ time; however, we need to shift the elements of the array to the right by 1 position. Therefore, the overall worst-case running time is still $\Theta (n^2)$.

2.3-7
Describe a $\Theta$ (n lg n)-time algorithm that, given a set S of n integers and another integer x, determines whether or not there exist two elements in S whose sum is exactly x.

Answer:
\begin{algorithm}
\begin{algorithmic}[1]
\Procedure{FIND-TWO-ELEMENTS-SUM-X}{S, n, x} \Comment{S has n integers}
    \State int B[1 ... 2n]
    \For{i = 1 to n}
        \State B[i] = x - S[i]
    \EndFor
    \For{i = n+1 to 2n}
        \State B[i] = S[i-n]
    \EndFor
    \State MERGE-SORT(B, 1, 2n)
    \For{i = 1 to 2n - 1}
        \If{B[i] == B[i+1]}
            \Return {true}
        \EndIf
    \EndFor
    \Return {false}
\EndProcedure
\end{algorithmic}
\end{algorithm}  
\end{document}
