\documentclass{article}
\title{Chapter 1 Exercises}
\author{Ziyang Jin}
\date{May 2016}
\setlength{\parindent}{0em}
\setlength{\parskip}{1em}

\begin{document}
1.2-2
Suppose we are comparing implementations of insertion sort and merge sort on the same machine. For inputs of size $n$, insertion sort runs in $8n^2$ steps, while merge sort runs in $64n(lg n)$ steps. For which values of n does insertion sort beat merge sort?

Answer:
  $$ 8n^2 = 64n (lg n) $$
  $$ n - 8(lgn) = 0 $$
  
  $$ n = 43:  n - 8(lg n) = -0.410118 $$
  $$ n = 44:  n - 8(lg n) =  0.324547 $$
  Therefore, when $1 <= n <= 43$, insertion sort beats merge sort.\par

1.2-3
What is the smallest value of $n$ such that an algorithm whose running time is $100n^2$ runs faster than an algorithm whose running time is $2^n$ on the same machine?

Answer:
  $$ 100n^2 = 2^n $$
  $$ 100n^2 - 2^n = 0 $$
  
  $$ n = 14: 100n^2 - 2^n = 3216 $$
  $$ n = 15: 100n^2 - 2^n = -10268 $$
  Therefore, the smallest value of n is 15.
\end{document}
